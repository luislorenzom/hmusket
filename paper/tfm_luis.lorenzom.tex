\documentclass[conference]{IEEEtran}
\IEEEoverridecommandlockouts
% The preceding line is only needed to identify funding in the first footnote. If that is unneeded, please comment it out.
\usepackage{cite}
\usepackage{amsmath,amssymb,amsfonts}
\usepackage{algorithmic}
\usepackage{graphicx}
\usepackage{textcomp}
\usepackage{xcolor}

\def\BibTeX{{\rm B\kern-.05em{\sc i\kern-.025em b}\kern-.08em
    T\kern-.1667em\lower.7ex\hbox{E}\kern-.125emX}}
\begin{document}

\title{HMusket: corrector de secuencias mediante el espectro k-mer basado en Hadoop}

\author{\IEEEauthorblockN{1\textsuperscript{st} Luis Lorenzo Mosquera}
\IEEEauthorblockA{\textit{Dpto. de ingeniería de computadores} \\
\textit{GAC (Grupo de Arquitectura de Computadores)}\\
A Coru\~na, Spain \\
luis.lorenzom@udc.es}
\and
\IEEEauthorblockN{2\textsuperscript{nd} Roberto Rey Exposito}
\IEEEauthorblockA{\textit{Dpto. de ingeniería de computadores} \\
\textit{GAC (Grupo de Arquitectura de Computadores)}\\
A Coru\~na, Spain \\
rreye@udc.es}
\and
\IEEEauthorblockN{3\textsuperscript{rd} Jorge González Domínguez}
\IEEEauthorblockA{\textit{Dpto. de ingeniería de computadores} \\
\textit{GAC (Grupo de Arquitectura de Computadores)}\\
A Coru\~na, Spain \\
jgonzaled@udc.es}
}

\maketitle

\begin{abstract}
This document is a model and instructions for \LaTeX.
This and the IEEEtran.cls file define the components of your paper [title, text, heads, etc.]. *CRITICAL: Do Not Use Symbols, Special Characters, Footnotes, 
or Math in Paper Title or Abstract.
\end{abstract}

\begin{IEEEkeywords}
Big Data, Hadoop, Map-Reduce, k-mer, sequence corrector
\end{IEEEkeywords}

\section{Introducción}
Debido a la aparición de la tecnologías conocidas como \textit{Next Generation Sequence} (NGS) se han obtenido grandes volúmenes de datos genéticos procedentes de diversos seres vivos (humanos, animales, plantas, etc). Estos vastos conjuntos de datos se utilizan principalmente para el estudio de los seres vivos secuenciados. 
No obstante con el abaratamiento de estas tecnologías acercan a los científicos hacia nuevos objetivos y metas donde tienen cabida todos estos datos.
\\

Además del ya citado estudio de los seres vivos, otros grandes objetivos o campos de estudios en el área de las ciencias de la vida (biología, medicina, etc.) es la predicción de enfermedades, de carácter o predisposición genética, en un estadío temprano para evitar futuras complicaciones en el paciente. \\
La evasión del fraude en productos alimenticios, tanto para detectar la procedencia u origen del producto en cuestión es otro de las grandes áreas de estudio principalmente por el tema comercial.\\
Por último, la interacción de la información contenida en los datos obtenidos durante la fase de secuenciación junto con una base de conocimiento tanto de las áreas de la farmacología y patología da como resultado la farmacogenética, pudiendo predecir qué fármaco es más efectivo, ideal o quizás poder diseñar un fármaco para un individuo.
\\

No obstante para llegar a tales fines y obtener unos resultados adecuados son necesarios una serie de pasos previos. Además de obtener una muestra de ADN y secuenciarla es necesario corregir ese conglomerado de datos, ya que durante la fase de amplificación/síntesis de nuevas copias por PCR es posible que se incorporen errores debido a fallos de ADN polimerasa.\\ Afortunadamente esta clase de fallos siguen un proceso estocástico y pueden ser corregidas por medio de diversos algoritmos.
\\

Debido a la problemática indicando en el párrafo superior junto con la gran demanda de datos genéticos que se están sucediendo las soluciones tradicionales para corregir los errores generados durante la fase de amplificación producen un cuello de botella a la hora de emplear esos datos ya que la mayoría de estas soluciones son secuenciales o de memoria compartida no logran reducir considerablemente los tiempos de pre-procesado de los datos.
\\

En el propósito de este trabajo es de proveer a los científicos una herramienta de memoria distribuida que pueda reducir el tiempo de corrección de las secuencias, además de esto, en este trabajo se realiza una análisis de las herramientas que hay disponibles en el mercado (tanto de memoria compartida como distribuida) una introducción y explicación de las tecnologías utilizadas a lo largo del proyecto, para posteriormente detallar el diseño y la implementación del software aquí expuesto, y finalizando con una serie de resultados donde se muestran una comparativa de tiempos y unas conclusiones

\section{Trabajos relacionados}

A lo largo de esta sección se presentarán las diferentes soluciones que hay actualmente en el mercado para la tarea de corregir las secuencias además de indicar tanto sus bondades y sus defectos.

\subsection{Memoria compartida}

Dentro del grupo de soluciones de memoria compartida cabe destacar las soluciones basada en GPU como son CUDE-EC y DecGPU las cuales utilizan un algoritmo basado en efectuar lecturas cortas del genoma lo que no provee una solución completa de errores ni una alta precisión en los resultados, no obstante al estar desarrolladas para GPUs presentan una índice de escalabilidad alto.\\

Otra solución que utiliza un algoritmo basado en lecturas cortas del genoma usando en este caso CPU en lugar de GPU es SOAP corrector. Posteriores versiones del algoritmo utilizan para determinadas operaciones un método basado en el grafo De Brujin, lo cual reduce drásticamente el uso de memoria en casos casos donde la longitud del genoma pueda presentar problemas.\\

Siguiendo con las soluciones que utilizan o están basadas en modelos de grafos, se encuentra Reptile, este software hace uso de un grafo Hamming para resolver las posibles ambigüedades que se encuentren en el genoma o región genómica a corregir, muy útil en casos que presenten errores de translocación.\\

Sin ser un software basado en un modelo de grafos, SGA consigue optimizar el uso de la memoria utilizando la tranformada de Burrows-Wheeler y el FM-Index para representar el espectro k-mer de la región genómica.\\

Además de los métodos de lecturas cortas y métodos basados en grafos hay alternativas basadas en modelos probabilísticos como por ejemplo: Quake, este corrector utiliza la probabilidad acumulada de los k-mer que conforman el genoma para poder clasificar si se trata de un error o no.\\

También existen soluciones que combinan dos posibles como por ejemplo: Hammer que se compone de una solución basada en un grafo Hamming y un modelo probabilístico de la probabilidad acumulada de los k-mer, al igual que hace Quake.\\

Por último indicar correctores que hacen uso de arrays de sufijos como son HiTEC, el cual instancia el genoma con distintos k-mers para posteriormente crear esos arrays y analizar los errores, o SHREC que utiliza un método parecido por HiTEC pero para la detección de indels y sustituciones.\\

\subsection{Memoria distribuida}

The IEEEtran class file is used to format your paper and style the text. All margins, 
column widths, line spaces, and text fonts are prescribed; please do not 
alter them. You may note peculiarities. For example, the head margin
measures proportionately more than is customary. This measurement 
and others are deliberate, using specifications that anticipate your paper 
as one part of the entire proceedings, and not as an independent document. 
Please do not revise any of the current designations.\\	

Quake\\

Cloud-CRS

%%%%%%%%%%%%%%%%%%%%%%%%%%%%%%%%%%%%%%%%%%%%%
%%%%%%%%%%%% TODO %%%%%%%%%%%%%%%%%%%%%%%%%%%%
%%%%%%%%%%%%%%%%%%%%%%%%%%%%%%%%%%%%%%%%%%%%%

\section{Conocimiento previo}

A lo largo de esta sección se expondrán y detallaran brevemente las diversas tecnologías utilizadas en este proyecto.

\subsection{Hadoop}
Framework, de código abierto, desarrollado en Java el cual esta orientado a la ejecución de aplicaciones distribuidas en un entorno clúster y al procesamiento de forma eficiente de grandes conjuntos de datos. Proveyendo de un ecosistema que permite a otras aplicaciones comunicarse entre si, hacer uso del mismo sistema de recursos (YARN) o de compartir el sistema de ficheros distribuido (HDFS).

\subsection{Map-Reduce}
Modelo de programación que surge ante la necesidad de procesar cantidades ingentes de datos. Este paradigma consta de dos operaciones.\\
%Introducir un ejemplo de codigo map
La operación Map convierte un par (clave, valor) en otro conjunto intermedio de datos en el mismo formato de tupla. Este formato de tupla hace mucho mas eficiente el procesado de los datos y una futura reconstrucción de los mismos.\\
La operación Reduce utiliza los conjuntos de datos intermedios generados por las operaciones Map para agruparlos y mostrar un resultado final.
%Introdcuri un ejemplo de codigo reduce

\subsection{HDFS}
Sistema de ficheros distribuido que permite ejecutar las aplicaciones distribuidas del cluster con una alta tolerancia a fallos 

\subsection{HSP}
Define abbreviations and acronyms the first time they are used in the text, 
even after they have been defined in the abstract. Abbreviations such as 
IEEE, SI, MKS, CGS, ac, dc, and rms do not have to be defined. Do not use 
abbreviations in the title or heads unless they are unavoidable.

\subsection{JNI}
Define abbreviations and acronyms the first time they are used in the text, 
even after they have been defined in the abstract. Abbreviations such as 
IEEE, SI, MKS, CGS, ac, dc, and rms do not have to be defined. Do not use 
abbreviations in the title or heads unless they are unavoidable.


\subsection{Musket}
Define abbreviations and acronyms the first time they are used in the text, 
even after they have been defined in the abstract. Abbreviations such as 
IEEE, SI, MKS, CGS, ac, dc, and rms do not have to be defined. Do not use 
abbreviations in the title or heads unless they are unavoidable.

\subsection{Some Common Mistakes}\label{SCM}
\begin{itemize}
\item The word ``data'' is plural, not singular.
\item The subscript for the permeability of vacuum $\mu_{0}$, and other common scientific constants, is zero with subscript formatting, not a lowercase letter ``o''.
\item In American English, commas, semicolons, periods, question and exclamation marks are located within quotation marks only when a complete thought or name is cited, such as a title or full quotation. When quotation marks are used, instead of a bold or italic typeface, to highlight a word or phrase, punctuation should appear outside of the quotation marks. A parenthetical phrase or statement at the end of a sentence is punctuated outside of the closing parenthesis (like this). (A parenthetical sentence is punctuated within the parentheses.)
\item A graph within a graph is an ``inset'', not an ``insert''. The word alternatively is preferred to the word ``alternately'' (unless you really mean something that alternates).
\item Do not use the word ``essentially'' to mean ``approximately'' or ``effectively''.
\item In your paper title, if the words ``that uses'' can accurately replace the word ``using'', capitalize the ``u''; if not, keep using lower-cased.
\item Be aware of the different meanings of the homophones ``affect'' and ``effect'', ``complement'' and ``compliment'', ``discreet'' and ``discrete'', ``principal'' and ``principle''.
\item Do not confuse ``imply'' and ``infer''.
\item The prefix ``non'' is not a word; it should be joined to the word it modifies, usually without a hyphen.
\item There is no period after the ``et'' in the Latin abbreviation ``et al.''.
\item The abbreviation ``i.e.'' means ``that is'', and the abbreviation ``e.g.'' means ``for example''.
\end{itemize}
An excellent style manual for science writers is \cite{b7}.

\subsection{Authors and Affiliations}
\textbf{The class file is designed for, but not limited to, six authors.} A 
minimum of one author is required for all conference articles. Author names 
should be listed starting from left to right and then moving down to the 
next line. This is the author sequence that will be used in future citations 
and by indexing services. Names should not be listed in columns nor group by 
affiliation. Please keep your affiliations as succinct as possible (for 
example, do not differentiate among departments of the same organization).

\subsection{Identify the Headings}
Headings, or heads, are organizational devices that guide the reader through 
your paper. There are two types: component heads and text heads.

Component heads identify the different components of your paper and are not 
topically subordinate to each other. Examples include Acknowledgments and 
References and, for these, the correct style to use is ``Heading 5''. Use 
``figure caption'' for your Figure captions, and ``table head'' for your 
table title. Run-in heads, such as ``Abstract'', will require you to apply a 
style (in this case, italic) in addition to the style provided by the drop 
down menu to differentiate the head from the text.

Text heads organize the topics on a relational, hierarchical basis. For 
example, the paper title is the primary text head because all subsequent 
material relates and elaborates on this one topic. If there are two or more 
sub-topics, the next level head (uppercase Roman numerals) should be used 
and, conversely, if there are not at least two sub-topics, then no subheads 
should be introduced.

\subsection{Figures and Tables}
\paragraph{Positioning Figures and Tables} Place figures and tables at the top and 
bottom of columns. Avoid placing them in the middle of columns. Large 
figures and tables may span across both columns. Figure captions should be 
below the figures; table heads should appear above the tables. Insert 
figures and tables after they are cited in the text. Use the abbreviation 
``Fig.~\ref{fig}'', even at the beginning of a sentence.

\begin{table}[htbp]
\caption{Table Type Styles}
\begin{center}
\begin{tabular}{|c|c|c|c|}
\hline
\textbf{Table}&\multicolumn{3}{|c|}{\textbf{Table Column Head}} \\
\cline{2-4} 
\textbf{Head} & \textbf{\textit{Table column subhead}}& \textbf{\textit{Subhead}}& \textbf{\textit{Subhead}} \\
\hline
copy& More table copy$^{\mathrm{a}}$& &  \\
\hline
\multicolumn{4}{l}{$^{\mathrm{a}}$Sample of a Table footnote.}
\end{tabular}
\label{tab1}
\end{center}
\end{table}

\begin{figure}[htbp]
%\centerline{\includegraphics{fig1.png}}
\caption{Example of a figure caption.}
\label{fig}
\end{figure}

Figure Labels: Use 8 point Times New Roman for Figure labels. Use words 
rather than symbols or abbreviations when writing Figure axis labels to 
avoid confusing the reader. As an example, write the quantity 
``Magnetization'', or ``Magnetization, M'', not just ``M''. If including 
units in the label, present them within parentheses. Do not label axes only 
with units. In the example, write ``Magnetization (A/m)'' or ``Magnetization 
\{A[m(1)]\}'', not just ``A/m''. Do not label axes with a ratio of 
quantities and units. For example, write ``Temperature (K)'', not 
``Temperature/K''.

\section*{Acknowledgment}

The preferred spelling of the word ``acknowledgment'' in America is without 
an ``e'' after the ``g''. Avoid the stilted expression ``one of us (R. B. 
G.) thanks $\ldots$''. Instead, try ``R. B. G. thanks$\ldots$''. Put sponsor 
acknowledgments in the unnumbered footnote on the first page.

\section*{References}

Please number citations consecutively within brackets \cite{b1}. The 
sentence punctuation follows the bracket \cite{b2}. Refer simply to the reference 
number, as in \cite{b3}---do not use ``Ref. \cite{b3}'' or ``reference \cite{b3}'' except at 
the beginning of a sentence: ``Reference \cite{b3} was the first $\ldots$''

Number footnotes separately in superscripts. Place the actual footnote at 
the bottom of the column in which it was cited. Do not put footnotes in the 
abstract or reference list. Use letters for table footnotes.

Unless there are six authors or more give all authors' names; do not use 
``et al.''. Papers that have not been published, even if they have been 
submitted for publication, should be cited as ``unpublished'' \cite{b4}. Papers 
that have been accepted for publication should be cited as ``in press'' \cite{b5}. 
Capitalize only the first word in a paper title, except for proper nouns and 
element symbols.

For papers published in translation journals, please give the English 
citation first, followed by the original foreign-language citation \cite{b6}.

\begin{thebibliography}{00}
\bibitem{b1} G. Eason, B. Noble, and I. N. Sneddon, ``On certain integrals of Lipschitz-Hankel type involving products of Bessel functions,'' Phil. Trans. Roy. Soc. London, vol. A247, pp. 529--551, April 1955.
\bibitem{b2} J. Clerk Maxwell, A Treatise on Electricity and Magnetism, 3rd ed., vol. 2. Oxford: Clarendon, 1892, pp.68--73.
\bibitem{b3} I. S. Jacobs and C. P. Bean, ``Fine particles, thin films and exchange anisotropy,'' in Magnetism, vol. III, G. T. Rado and H. Suhl, Eds. New York: Academic, 1963, pp. 271--350.
\bibitem{b4} K. Elissa, ``Title of paper if known,'' unpublished.
\bibitem{b5} R. Nicole, ``Title of paper with only first word capitalized,'' J. Name Stand. Abbrev., in press.
\bibitem{b6} Y. Yorozu, M. Hirano, K. Oka, and Y. Tagawa, ``Electron spectroscopy studies on magneto-optical media and plastic substrate interface,'' IEEE Transl. J. Magn. Japan, vol. 2, pp. 740--741, August 1987 [Digests 9th Annual Conf. Magnetics Japan, p. 301, 1982].
\bibitem{b7} M. Young, The Technical Writer's Handbook. Mill Valley, CA: University Science, 1989.
\end{thebibliography}
\vspace{12pt}
\color{red}
IEEE conference templates contain guidance text for composing and formatting conference papers. Please ensure that all template text is removed from your conference paper prior to submission to the conference. Failure to remove the template text from your paper may result in your paper not being published.

\end{document}
